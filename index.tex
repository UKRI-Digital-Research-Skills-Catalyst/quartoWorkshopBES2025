% Options for packages loaded elsewhere
\PassOptionsToPackage{unicode}{hyperref}
\PassOptionsToPackage{hyphens}{url}
\PassOptionsToPackage{dvipsnames,svgnames,x11names}{xcolor}
%
\documentclass[
  letterpaper,
  DIV=11,
  numbers=noendperiod]{scrartcl}

\usepackage{amsmath,amssymb}
\usepackage{iftex}
\ifPDFTeX
  \usepackage[T1]{fontenc}
  \usepackage[utf8]{inputenc}
  \usepackage{textcomp} % provide euro and other symbols
\else % if luatex or xetex
  \usepackage{unicode-math}
  \defaultfontfeatures{Scale=MatchLowercase}
  \defaultfontfeatures[\rmfamily]{Ligatures=TeX,Scale=1}
\fi
\usepackage{lmodern}
\ifPDFTeX\else  
    % xetex/luatex font selection
\fi
% Use upquote if available, for straight quotes in verbatim environments
\IfFileExists{upquote.sty}{\usepackage{upquote}}{}
\IfFileExists{microtype.sty}{% use microtype if available
  \usepackage[]{microtype}
  \UseMicrotypeSet[protrusion]{basicmath} % disable protrusion for tt fonts
}{}
\makeatletter
\@ifundefined{KOMAClassName}{% if non-KOMA class
  \IfFileExists{parskip.sty}{%
    \usepackage{parskip}
  }{% else
    \setlength{\parindent}{0pt}
    \setlength{\parskip}{6pt plus 2pt minus 1pt}}
}{% if KOMA class
  \KOMAoptions{parskip=half}}
\makeatother
\usepackage{xcolor}
\setlength{\emergencystretch}{3em} % prevent overfull lines
\setcounter{secnumdepth}{-\maxdimen} % remove section numbering
% Make \paragraph and \subparagraph free-standing
\makeatletter
\ifx\paragraph\undefined\else
  \let\oldparagraph\paragraph
  \renewcommand{\paragraph}{
    \@ifstar
      \xxxParagraphStar
      \xxxParagraphNoStar
  }
  \newcommand{\xxxParagraphStar}[1]{\oldparagraph*{#1}\mbox{}}
  \newcommand{\xxxParagraphNoStar}[1]{\oldparagraph{#1}\mbox{}}
\fi
\ifx\subparagraph\undefined\else
  \let\oldsubparagraph\subparagraph
  \renewcommand{\subparagraph}{
    \@ifstar
      \xxxSubParagraphStar
      \xxxSubParagraphNoStar
  }
  \newcommand{\xxxSubParagraphStar}[1]{\oldsubparagraph*{#1}\mbox{}}
  \newcommand{\xxxSubParagraphNoStar}[1]{\oldsubparagraph{#1}\mbox{}}
\fi
\makeatother

\usepackage{color}
\usepackage{fancyvrb}
\newcommand{\VerbBar}{|}
\newcommand{\VERB}{\Verb[commandchars=\\\{\}]}
\DefineVerbatimEnvironment{Highlighting}{Verbatim}{commandchars=\\\{\}}
% Add ',fontsize=\small' for more characters per line
\usepackage{framed}
\definecolor{shadecolor}{RGB}{241,243,245}
\newenvironment{Shaded}{\begin{snugshade}}{\end{snugshade}}
\newcommand{\AlertTok}[1]{\textcolor[rgb]{0.68,0.00,0.00}{#1}}
\newcommand{\AnnotationTok}[1]{\textcolor[rgb]{0.37,0.37,0.37}{#1}}
\newcommand{\AttributeTok}[1]{\textcolor[rgb]{0.40,0.45,0.13}{#1}}
\newcommand{\BaseNTok}[1]{\textcolor[rgb]{0.68,0.00,0.00}{#1}}
\newcommand{\BuiltInTok}[1]{\textcolor[rgb]{0.00,0.23,0.31}{#1}}
\newcommand{\CharTok}[1]{\textcolor[rgb]{0.13,0.47,0.30}{#1}}
\newcommand{\CommentTok}[1]{\textcolor[rgb]{0.37,0.37,0.37}{#1}}
\newcommand{\CommentVarTok}[1]{\textcolor[rgb]{0.37,0.37,0.37}{\textit{#1}}}
\newcommand{\ConstantTok}[1]{\textcolor[rgb]{0.56,0.35,0.01}{#1}}
\newcommand{\ControlFlowTok}[1]{\textcolor[rgb]{0.00,0.23,0.31}{\textbf{#1}}}
\newcommand{\DataTypeTok}[1]{\textcolor[rgb]{0.68,0.00,0.00}{#1}}
\newcommand{\DecValTok}[1]{\textcolor[rgb]{0.68,0.00,0.00}{#1}}
\newcommand{\DocumentationTok}[1]{\textcolor[rgb]{0.37,0.37,0.37}{\textit{#1}}}
\newcommand{\ErrorTok}[1]{\textcolor[rgb]{0.68,0.00,0.00}{#1}}
\newcommand{\ExtensionTok}[1]{\textcolor[rgb]{0.00,0.23,0.31}{#1}}
\newcommand{\FloatTok}[1]{\textcolor[rgb]{0.68,0.00,0.00}{#1}}
\newcommand{\FunctionTok}[1]{\textcolor[rgb]{0.28,0.35,0.67}{#1}}
\newcommand{\ImportTok}[1]{\textcolor[rgb]{0.00,0.46,0.62}{#1}}
\newcommand{\InformationTok}[1]{\textcolor[rgb]{0.37,0.37,0.37}{#1}}
\newcommand{\KeywordTok}[1]{\textcolor[rgb]{0.00,0.23,0.31}{\textbf{#1}}}
\newcommand{\NormalTok}[1]{\textcolor[rgb]{0.00,0.23,0.31}{#1}}
\newcommand{\OperatorTok}[1]{\textcolor[rgb]{0.37,0.37,0.37}{#1}}
\newcommand{\OtherTok}[1]{\textcolor[rgb]{0.00,0.23,0.31}{#1}}
\newcommand{\PreprocessorTok}[1]{\textcolor[rgb]{0.68,0.00,0.00}{#1}}
\newcommand{\RegionMarkerTok}[1]{\textcolor[rgb]{0.00,0.23,0.31}{#1}}
\newcommand{\SpecialCharTok}[1]{\textcolor[rgb]{0.37,0.37,0.37}{#1}}
\newcommand{\SpecialStringTok}[1]{\textcolor[rgb]{0.13,0.47,0.30}{#1}}
\newcommand{\StringTok}[1]{\textcolor[rgb]{0.13,0.47,0.30}{#1}}
\newcommand{\VariableTok}[1]{\textcolor[rgb]{0.07,0.07,0.07}{#1}}
\newcommand{\VerbatimStringTok}[1]{\textcolor[rgb]{0.13,0.47,0.30}{#1}}
\newcommand{\WarningTok}[1]{\textcolor[rgb]{0.37,0.37,0.37}{\textit{#1}}}

\providecommand{\tightlist}{%
  \setlength{\itemsep}{0pt}\setlength{\parskip}{0pt}}\usepackage{longtable,booktabs,array}
\usepackage{calc} % for calculating minipage widths
% Correct order of tables after \paragraph or \subparagraph
\usepackage{etoolbox}
\makeatletter
\patchcmd\longtable{\par}{\if@noskipsec\mbox{}\fi\par}{}{}
\makeatother
% Allow footnotes in longtable head/foot
\IfFileExists{footnotehyper.sty}{\usepackage{footnotehyper}}{\usepackage{footnote}}
\makesavenoteenv{longtable}
\usepackage{graphicx}
\makeatletter
\newsavebox\pandoc@box
\newcommand*\pandocbounded[1]{% scales image to fit in text height/width
  \sbox\pandoc@box{#1}%
  \Gscale@div\@tempa{\textheight}{\dimexpr\ht\pandoc@box+\dp\pandoc@box\relax}%
  \Gscale@div\@tempb{\linewidth}{\wd\pandoc@box}%
  \ifdim\@tempb\p@<\@tempa\p@\let\@tempa\@tempb\fi% select the smaller of both
  \ifdim\@tempa\p@<\p@\scalebox{\@tempa}{\usebox\pandoc@box}%
  \else\usebox{\pandoc@box}%
  \fi%
}
% Set default figure placement to htbp
\def\fps@figure{htbp}
\makeatother
% definitions for citeproc citations
\NewDocumentCommand\citeproctext{}{}
\NewDocumentCommand\citeproc{mm}{%
  \begingroup\def\citeproctext{#2}\cite{#1}\endgroup}
\makeatletter
 % allow citations to break across lines
 \let\@cite@ofmt\@firstofone
 % avoid brackets around text for \cite:
 \def\@biblabel#1{}
 \def\@cite#1#2{{#1\if@tempswa , #2\fi}}
\makeatother
\newlength{\cslhangindent}
\setlength{\cslhangindent}{1.5em}
\newlength{\csllabelwidth}
\setlength{\csllabelwidth}{3em}
\newenvironment{CSLReferences}[2] % #1 hanging-indent, #2 entry-spacing
 {\begin{list}{}{%
  \setlength{\itemindent}{0pt}
  \setlength{\leftmargin}{0pt}
  \setlength{\parsep}{0pt}
  % turn on hanging indent if param 1 is 1
  \ifodd #1
   \setlength{\leftmargin}{\cslhangindent}
   \setlength{\itemindent}{-1\cslhangindent}
  \fi
  % set entry spacing
  \setlength{\itemsep}{#2\baselineskip}}}
 {\end{list}}
\usepackage{calc}
\newcommand{\CSLBlock}[1]{\hfill\break\parbox[t]{\linewidth}{\strut\ignorespaces#1\strut}}
\newcommand{\CSLLeftMargin}[1]{\parbox[t]{\csllabelwidth}{\strut#1\strut}}
\newcommand{\CSLRightInline}[1]{\parbox[t]{\linewidth - \csllabelwidth}{\strut#1\strut}}
\newcommand{\CSLIndent}[1]{\hspace{\cslhangindent}#1}

\KOMAoption{captions}{tableheading}
\makeatletter
\@ifpackageloaded{caption}{}{\usepackage{caption}}
\AtBeginDocument{%
\ifdefined\contentsname
  \renewcommand*\contentsname{Table of contents}
\else
  \newcommand\contentsname{Table of contents}
\fi
\ifdefined\listfigurename
  \renewcommand*\listfigurename{List of Figures}
\else
  \newcommand\listfigurename{List of Figures}
\fi
\ifdefined\listtablename
  \renewcommand*\listtablename{List of Tables}
\else
  \newcommand\listtablename{List of Tables}
\fi
\ifdefined\figurename
  \renewcommand*\figurename{Figure}
\else
  \newcommand\figurename{Figure}
\fi
\ifdefined\tablename
  \renewcommand*\tablename{Table}
\else
  \newcommand\tablename{Table}
\fi
}
\@ifpackageloaded{float}{}{\usepackage{float}}
\floatstyle{ruled}
\@ifundefined{c@chapter}{\newfloat{codelisting}{h}{lop}}{\newfloat{codelisting}{h}{lop}[chapter]}
\floatname{codelisting}{Listing}
\newcommand*\listoflistings{\listof{codelisting}{List of Listings}}
\makeatother
\makeatletter
\makeatother
\makeatletter
\@ifpackageloaded{caption}{}{\usepackage{caption}}
\@ifpackageloaded{subcaption}{}{\usepackage{subcaption}}
\makeatother

\usepackage{bookmark}

\IfFileExists{xurl.sty}{\usepackage{xurl}}{} % add URL line breaks if available
\urlstyle{same} % disable monospaced font for URLs
\hypersetup{
  pdftitle={Reproducible and Automated Reporting with Quarto},
  pdfauthor={Pasky Miranda (UKRI Digital Research Skills Catalyst)},
  colorlinks=true,
  linkcolor={blue},
  filecolor={Maroon},
  citecolor={Blue},
  urlcolor={Blue},
  pdfcreator={LaTeX via pandoc}}


\title{Reproducible and Automated Reporting with Quarto}
\author{Pasky Miranda (UKRI Digital Research Skills Catalyst)}
\date{}

\begin{document}
\maketitle


https://tinyurl.com/QuartoBES

\pandocbounded{\includegraphics[keepaspectratio]{images/catalystlogo.png}}

This workshop has been developed by the
\href{https://www.linkedin.com/company/ukri-digital-research-skills-catalyst/}{UKRI
Digital Research Skills Catalyst}.

\section{Housekeeping}\label{housekeeping}

This tutorial is available as a webpage here:
\href{https://ukri-digital-research-skills-catalyst.github.io/quartoWorkshopBES2025}{github}.\\
If you don't have R Studio installed or working, no problem! It's also
available as a positcloud instance:
\href{https://posit.cloud/content/11607987}{positcloud}.

If you see this emoji 🎬 then it's an instruction to do something.

\section{What You'll Learn Today}\label{what-youll-learn-today}

\begin{itemize}
\tightlist
\item
  Markdown basics
\item
  How to create a Quarto document
\item
  What a YAML does
\item
  Adding content to a Quarto document
\item
  How to render Inline code
\item
  How to render Figures + tables
\item
  How to use Auto-renumbering
\item
  How to use Visual editor
\item
  How to add Citations
\end{itemize}

\section{What is Quarto?}\label{what-is-quarto}

Quarto is a modern scientific publishing system for \textbf{dynamic,
reproducible documents}.

\begin{itemize}
\tightlist
\item
  Write text + code together
\item
  Render to \textbf{HTML}, \textbf{PDF}, \textbf{Word}, \textbf{slides},
  \textbf{manuscripts}, more
\item
  Supports R, Python, Julia, Observable
\item
  Like R Markdown, but more flexible + standardised
\end{itemize}

\section{Why Use Quarto?}\label{why-use-quarto}

Typical write-up workflow:

\begin{itemize}
\tightlist
\item
  Figures/tables manually updated
\item
  Graphs are saved to file then inserted manually
\item
  Cross-references break when ordering changes
\item
  Reformatting for journals takes hours
\item
  Lots of copying/pasting
\item
  Lots of potential for errors
\end{itemize}

With Quarto you don't have to worry about any of these. It does the work
for you!

\subsection{Publishing}\label{publishing}

To address the reproducibility crisis journals now increasingly want you
to have your code/workbook submitted with the manuscript. Working in
Quarto allows you have all of that all in one place, readily exportable
to whichever format the journal requires.

\subsection{Demo: Multiple Formats}\label{demo-multiple-formats}

These are all the \textbf{same} \texttt{.qmd} document, just with
different output formats:

\begin{itemize}
\tightlist
\item
  \href{index.html}{HTML}
\item
  \href{pdf.pdf}{PDF}
\item
  \href{docx.docx}{Word document}
\item
  \href{revealjs.html}{Revealjs slide deck}
\end{itemize}

(Some tidying up may be required for certain formats, but it all can run
from the same source document!)

\section{Markdown Basics}\label{markdown-basics}

\begin{itemize}
\tightlist
\item
  The YAML header sets the default behaviour for the document and is
  between \texttt{-\/-\/-} at the top of the document (more on this
  later)
\item
  R Code chunks are between
  \texttt{\textasciigrave{}\textasciigrave{}\textasciigrave{}\{r\}} and
  \texttt{\textasciigrave{}\textasciigrave{}\textasciigrave{}} and chunk
  options, starting \texttt{\#\textbar{}}, determine how/whether they
  run whether code/output is included in the rendered document
\item
  You can run code chunk interactively or through rendering
\item
  Comments: \texttt{\textbar{}\#}, \texttt{\#} in code chunks,
  \texttt{\textless{}!-\/-\ in\ text\ -\/-\textgreater{}} but use
  Ctrl+Shift+C
\end{itemize}

\subsection{Text Formatting}\label{text-formatting}

\begin{longtable}[]{@{}
  >{\raggedright\arraybackslash}p{(\linewidth - 2\tabcolsep) * \real{0.2917}}
  >{\raggedright\arraybackslash}p{(\linewidth - 2\tabcolsep) * \real{0.3194}}@{}}
\toprule\noalign{}
\begin{minipage}[b]{\linewidth}\raggedright
Markdown Syntax
\end{minipage} & \begin{minipage}[b]{\linewidth}\raggedright
Output
\end{minipage} \\
\midrule\noalign{}
\endhead
\bottomrule\noalign{}
\endlastfoot
\begin{minipage}[t]{\linewidth}\raggedright
\begin{Shaded}
\begin{Highlighting}[]
\NormalTok{*italics*,**bold**}
\end{Highlighting}
\end{Shaded}
\end{minipage} & \emph{italics}, \textbf{bold} \\
\begin{minipage}[t]{\linewidth}\raggedright
\begin{Shaded}
\begin{Highlighting}[]
\NormalTok{***bold italics***}
\end{Highlighting}
\end{Shaded}
\end{minipage} & \textbf{\emph{bold italics}} \\
\begin{minipage}[t]{\linewidth}\raggedright
\begin{Shaded}
\begin{Highlighting}[]
\NormalTok{superscript\^{}2\^{}}
\end{Highlighting}
\end{Shaded}
\end{minipage} & superscript\textsuperscript{2} \\
\begin{minipage}[t]{\linewidth}\raggedright
\begin{Shaded}
\begin{Highlighting}[]
\NormalTok{subscript\textasciitilde{}2\textasciitilde{}}
\end{Highlighting}
\end{Shaded}
\end{minipage} & subscript\textsubscript{2} \\
\begin{minipage}[t]{\linewidth}\raggedright
\begin{Shaded}
\begin{Highlighting}[]
\NormalTok{\textasciitilde{}\textasciitilde{}strikethrough\textasciitilde{}\textasciitilde{}}
\end{Highlighting}
\end{Shaded}
\end{minipage} & \textasciitilde{}
\textsubscript{strikethrough}\textasciitilde{} \\
\begin{minipage}[t]{\linewidth}\raggedright
\begin{Shaded}
\begin{Highlighting}[]
\InformationTok{\textasciigrave{}verbatim code\textasciigrave{}}
\end{Highlighting}
\end{Shaded}
\end{minipage} & \texttt{verbatim\ code} \\
\end{longtable}

\subsection{Headings}\label{headings}

\begin{longtable}[]{@{}
  >{\raggedright\arraybackslash}p{(\linewidth - 2\tabcolsep) * \real{0.2500}}
  >{\raggedright\arraybackslash}p{(\linewidth - 2\tabcolsep) * \real{0.1944}}@{}}
\toprule\noalign{}
\begin{minipage}[b]{\linewidth}\raggedright
Markdown Syntax
\end{minipage} & \begin{minipage}[b]{\linewidth}\raggedright
Output
\end{minipage} \\
\midrule\noalign{}
\endhead
\bottomrule\noalign{}
\endlastfoot
\begin{minipage}[t]{\linewidth}\raggedright
\begin{Shaded}
\begin{Highlighting}[]
\FunctionTok{\# Header 1}
\end{Highlighting}
\end{Shaded}
\end{minipage} & \begin{minipage}[t]{\linewidth}\raggedright
\section{Header 1}\label{header-1}
\end{minipage} \\
\begin{minipage}[t]{\linewidth}\raggedright
\begin{Shaded}
\begin{Highlighting}[]
\FunctionTok{\#\# Header 2}
\end{Highlighting}
\end{Shaded}
\end{minipage} & \begin{minipage}[t]{\linewidth}\raggedright
\subsection{Header 2}\label{header-2}
\end{minipage} \\
\begin{minipage}[t]{\linewidth}\raggedright
\begin{Shaded}
\begin{Highlighting}[]
\FunctionTok{\#\#\# Header 3}
\end{Highlighting}
\end{Shaded}
\end{minipage} & \begin{minipage}[t]{\linewidth}\raggedright
\section{}\label{section}

\subsection{Header 3}\label{header-3}
\end{minipage} \\
\begin{minipage}[t]{\linewidth}\raggedright
\begin{Shaded}
\begin{Highlighting}[]
\FunctionTok{\#\#\#\# Header 4}
\end{Highlighting}
\end{Shaded}
\end{minipage} & \begin{minipage}[t]{\linewidth}\raggedright
\section{}\label{section-1}

\section{}\label{section-2}

\subsection{Header 4}\label{header-4}
\end{minipage} \\
\end{longtable}

\subsection{Links \& Images}\label{links-images}

\begin{longtable}[]{@{}
  >{\raggedright\arraybackslash}p{(\linewidth - 2\tabcolsep) * \real{0.4306}}
  >{\raggedright\arraybackslash}p{(\linewidth - 2\tabcolsep) * \real{0.4306}}@{}}
\toprule\noalign{}
\begin{minipage}[b]{\linewidth}\raggedright
Markdown Syntax
\end{minipage} & \begin{minipage}[b]{\linewidth}\raggedright
Output
\end{minipage} \\
\midrule\noalign{}
\endhead
\bottomrule\noalign{}
\endlastfoot
\begin{minipage}[t]{\linewidth}\raggedright
\begin{Shaded}
\begin{Highlighting}[]
\CommentTok{[}\OtherTok{Quarto}\CommentTok{](https://quarto.org)}
\end{Highlighting}
\end{Shaded}
\end{minipage} & \href{https://quarto.org}{Quarto} \\
\begin{minipage}[t]{\linewidth}\raggedright
\begin{Shaded}
\begin{Highlighting}[]
\AlertTok{![Caption](images/chaff.jpg)}
\end{Highlighting}
\end{Shaded}
\end{minipage} & \begin{minipage}[t]{\linewidth}\raggedright
\begin{figure}[H]

{\centering \pandocbounded{\includegraphics[keepaspectratio]{images/chaff.jpg}}

}

\caption{Caption}

\end{figure}%
\end{minipage} \\
\end{longtable}

\subsection{Equations}\label{equations}

Use \texttt{\$} delimiters for inline maths and \texttt{\$\$} delimiters
for display maths. For example:

\begin{longtable}[]{@{}
  >{\raggedright\arraybackslash}p{(\linewidth - 2\tabcolsep) * \real{0.4444}}
  >{\raggedright\arraybackslash}p{(\linewidth - 2\tabcolsep) * \real{0.3750}}@{}}
\toprule\noalign{}
\begin{minipage}[b]{\linewidth}\raggedright
Markdown Syntax
\end{minipage} & \begin{minipage}[b]{\linewidth}\raggedright
Output
\end{minipage} \\
\midrule\noalign{}
\endhead
\bottomrule\noalign{}
\endlastfoot
\begin{minipage}[t]{\linewidth}\raggedright
\begin{Shaded}
\begin{Highlighting}[]
\AnnotationTok{inline maths:}\CommentTok{ $E = mc\^{}\{2\}$}
\end{Highlighting}
\end{Shaded}
\end{minipage} & inline maths: \(E=mc^{2}\) \\
\begin{minipage}[t]{\linewidth}\raggedright
\begin{Shaded}
\begin{Highlighting}[]
\AnnotationTok{display maths:}\CommentTok{ $$E = mc\^{}\{2\}$$}
\end{Highlighting}
\end{Shaded}
\end{minipage} & display maths:

\(E = mc^{2}\) \\
\end{longtable}

\section{Create a QMD Document}\label{create-a-qmd-document}

Let's create a Quarto Document (a \texttt{.qmd} file) in RStudio!

🎬 Go to File \textgreater{} New File \textgreater{} New Quarto
Document\\
🎬 Give your file a name\\
🎬 Choose Engine: Knitr\\
🎬 Turn off Visual Editor (for now! We'll learn to use Source editor
first)\\
🎬 Now hit Render (or Ctrl+Shift+K).\\

You'll see the Rendered default document in the Viewer panel.

\subsection{Some recommended RStudio
settings}\label{some-recommended-rstudio-settings}

🎬 Tools \textgreater{} Global Options\\
- General:\\
- Turn off the three ``Restore \ldots.'' options\\
- Turn ``Save workspace to .RData on exit'' to Never - R Markdown:\\
- Turn ``Show output preview in:'' to Viewer pane\\

\subsection{Edit the YAML}\label{edit-the-yaml}

As mentioned before, the YAML determines of the settings of the
\texttt{.qmd} document. It is defined between \texttt{-\/-\/-} at the
top of the document.

The default YAML header will look a bit like this:

\begin{Shaded}
\begin{Highlighting}[]
\PreprocessorTok{{-}{-}{-}}
\FunctionTok{title}\KeywordTok{:}\AttributeTok{ }\StringTok{"BES2025 Quarto"}
\FunctionTok{author}\KeywordTok{:}\AttributeTok{ }\StringTok{"Santa Claus"}
\FunctionTok{format}\KeywordTok{:}\AttributeTok{ html}
\FunctionTok{engine}\KeywordTok{:}\AttributeTok{ knitr}
\PreprocessorTok{{-}{-}{-}}
\end{Highlighting}
\end{Shaded}

🎬 Feel free to edit this. Try making your own title and add yourself as
the author.

Deafult code chunk options can also be determined in the YAML:

\begin{itemize}
\tightlist
\item
  \texttt{echo:\ false} code will not be included in output
\item
  \texttt{include:\ true} output will be included
\item
  \texttt{error:\ false} halt render if a code error occurs
\item
  \texttt{message:\ false} messages and warnings will not be included
\item
  \texttt{warning:\ false}
\end{itemize}

YAMLs can be saved as separate .YAML files and then pointed to by
individual Quarto documents like so:

\begin{Shaded}
\begin{Highlighting}[]
\PreprocessorTok{{-}{-}{-}}
\FunctionTok{metadata{-}files}\KeywordTok{:}
\AttributeTok{  }\KeywordTok{{-}}\AttributeTok{ settings.yaml}
\PreprocessorTok{{-}{-}{-}}
\end{Highlighting}
\end{Shaded}

Useful if you want the same formatting for multiple documents!

\subsection{Adding basic content}\label{adding-basic-content}

🎬 Add the code chunk below (Ctrl-Shift-I) for a simple graph, and then
hit Render (Ctrl-Shift-K)

\begin{Shaded}
\begin{Highlighting}[]
\InformationTok{\textasciigrave{}\textasciigrave{}\textasciigrave{}\{r\}}
\InformationTok{hist(rnorm(100))}
\InformationTok{\textasciigrave{}\textasciigrave{}\textasciigrave{}}
\end{Highlighting}
\end{Shaded}

\pandocbounded{\includegraphics[keepaspectratio]{index_files/figure-pdf/unnamed-chunk-1-1.pdf}}

Congratulations! You've created your first Quarto content!

\subsection{Loading in packages}\label{loading-in-packages}

Let's try loading in an R package, in this case tidyverse (Wickham et
al. 2019).

🎬 Add this code chunk for package loading:

\begin{Shaded}
\begin{Highlighting}[]
\InformationTok{\textasciigrave{}\textasciigrave{}\textasciigrave{}\{r\}}
\InformationTok{\#| label: load{-}packages}
\InformationTok{library(tidyverse)}
\InformationTok{\textasciigrave{}\textasciigrave{}\textasciigrave{}}
\end{Highlighting}
\end{Shaded}

\begin{verbatim}
-- Attaching core tidyverse packages ------------------------ tidyverse 2.0.0 --
v dplyr     1.1.4     v readr     2.1.5
v forcats   1.0.0     v stringr   1.5.1
v ggplot2   3.5.0     v tibble    3.2.1
v lubridate 1.9.3     v tidyr     1.3.1
v purrr     1.0.2     
-- Conflicts ------------------------------------------ tidyverse_conflicts() --
x dplyr::filter() masks stats::filter()
x dplyr::lag()    masks stats::lag()
i Use the conflicted package (<http://conflicted.r-lib.org/>) to force all conflicts to become errors
\end{verbatim}

\texttt{\#\textbar{}\ label:\ load-packages} is a code chunk label.
These are useful (here) and essential (later) for cross-referencing.

We can run individual code chunks in the R editor. This is useful for
troubleshooting R code without having to Render every time (which can be
lengthy with larger documents or datasets).

\subsection{Loading in Data}\label{loading-in-data}

Let's now try loading in some real data.

🎬 Make a folder called \texttt{data-raw}\\
🎬 Save \href{data-raw/chaff.txt}{chaff.txt} to \texttt{data-raw}\\
🎬 Add the following code chunk and run it:\\

\begin{Shaded}
\begin{Highlighting}[]
\InformationTok{\textasciigrave{}\textasciigrave{}\textasciigrave{}\{r\}}
\InformationTok{\#| label: import{-}data}
\InformationTok{file \textless{}{-} "data{-}raw/chaff.txt"}
\InformationTok{chaff \textless{}{-} read\_table(file)}
\InformationTok{\textasciigrave{}\textasciigrave{}\textasciigrave{}}
\end{Highlighting}
\end{Shaded}

\begin{verbatim}

-- Column specification --------------------------------------------------------
cols(
  subspecies = col_character(),
  mass = col_double()
)
\end{verbatim}

The data is loaded into R now. Let's now add a code chunk to summarise
the dataset.

\subsection{Data Summary}\label{data-summary}

🎬 Add this data summarisation chunk and run it

\begin{Shaded}
\begin{Highlighting}[]
\InformationTok{\textasciigrave{}\textasciigrave{}\textasciigrave{}\{r\}}
\InformationTok{\#| label: data{-}summary}
\InformationTok{chaff\_summary \textless{}{-} chaff |\textgreater{} }
\InformationTok{  group\_by(subspecies) |\textgreater{}  }
\InformationTok{  summarise(mean = mean(mass, na.rm = TRUE),}
\InformationTok{            sd = sd(mass, na.rm = TRUE),}
\InformationTok{            n = length(mass),}
\InformationTok{            se = sd / sqrt(n))}
\InformationTok{\textasciigrave{}\textasciigrave{}\textasciigrave{}}
\end{Highlighting}
\end{Shaded}

Our data is now summarised and ready to be reported in the text.

\subsection{Inline Code}\label{inline-code}

Usually results are ``hard coded'' into the text of a write-up, we
manually type them in.

If the dataset changes we'd have to go back to every instance of that
reporting and change them manually, which is laborious and error-prone
(how often have you mis-remembered or mis-typed a number?)

We can instead use inline code to automatically place data (such as
sample size, mean) into the text! Any code output can be placed as
inline code.

Inline code goes between `r` and ` .

For example by writing:

The squareroot of 2 is `r \texttt{sqrt(2)} `

you will get:

The squareroot of 2 is 1.4142136

🎬 To summarise our data we can add this code chunk to define the
variable \texttt{n\_coelebs}:

\begin{Shaded}
\begin{Highlighting}[]
\InformationTok{\textasciigrave{}\textasciigrave{}\textasciigrave{}\{r\}}
\InformationTok{n\_coelebs \textless{}{-} chaff\_summary$n[chaff\_summary$subspecies == "coelebs"]}
\InformationTok{\textasciigrave{}\textasciigrave{}\textasciigrave{}}
\end{Highlighting}
\end{Shaded}

🎬 And then add that to the text as follows:

We sampled `r \texttt{n\_coelebs}` *F. c.~coelebs*~males.

🎬 Then hit Render (Ctrl+Shift+K) and it will display as:

We sampled 20 \emph{F. c.~coelebs} males.

\section{Figures}\label{figures}

Now let's try and insert a graph.

🎬 Add a code chunk to the Results section with:

\begin{Shaded}
\begin{Highlighting}[]
\InformationTok{\textasciigrave{}\textasciigrave{}\textasciigrave{}\{r\}}
\InformationTok{\#| label: fig{-}chaff}
\InformationTok{\#| fig{-}cap: "Mass of Chaffinches"}
\InformationTok{ggplot() +}
\InformationTok{  geom\_point(data = chaff, aes(x = subspecies, y = mass),}
\InformationTok{             position = position\_jitter(width = 0.1, height = 0),}
\InformationTok{             colour = "gray50") +}
\InformationTok{  geom\_errorbar(data = chaff\_summary, }
\InformationTok{                aes(x = subspecies, ymin = mean {-} se, ymax = mean + se),}
\InformationTok{                width = 0.3) +}
\InformationTok{  geom\_errorbar(data = chaff\_summary, }
\InformationTok{                aes(x = subspecies, ymin = mean, ymax = mean),}
\InformationTok{                width = 0.2) +}
\InformationTok{  scale\_y\_continuous(name = "Mass (g)", }
\InformationTok{                     limits = c(0, 30), }
\InformationTok{                     expand = c(0, 0)) +}
\InformationTok{  scale\_x\_discrete(name = "Subspecies", }
\InformationTok{                   labels = c("Canariensis", "Coelebs")) +}
\InformationTok{  theme\_classic()}
\InformationTok{\textasciigrave{}\textasciigrave{}\textasciigrave{}}
\end{Highlighting}
\end{Shaded}

\begin{figure}[H]

\centering{

\pandocbounded{\includegraphics[keepaspectratio]{index_files/figure-pdf/fig-chaff-1.pdf}}

}

\caption{\label{fig-chaff}Mass of Chaffinches}

\end{figure}%

🎬 Render (Ctrl+Shift+K)

Note how it is automatically labelled as Figure 1! Quarto will
automatically number tables and figures for you.

\subsection{Cross-referencing Figures}\label{cross-referencing-figures}

Remember how we are labelling our code chunks?

The label of the code chunk is important when creating figures because
it is used to cross reference the figure in the text. We give cross
references with \texttt{@} followed by the label of the code chunk.

That is starts with a \texttt{fig-} prefix also matters - this is what
determines what is treated as a figure numbered in order of appearance.

🎬 If we add ``See \texttt{@fig-chaff}'' to our text and hit Render
it'll display as:

See Figure~\ref{fig-chaff}.

\subsection{Figure legends}\label{figure-legends}

There are two ways to add a legend:

\begin{enumerate}
\def\labelenumi{\arabic{enumi}.}
\tightlist
\item
  Add the \texttt{fig-cap} chunk option:
  \texttt{fig-cap:\ "Mass\ of\ chaffinches"} to the code chunk.
\item
  Use a div. \texttt{:::}
\end{enumerate}

Divs are more flexible:

\begin{itemize}
\tightlist
\item
  You can include inline code, citations, special characters and
  formatting
\item
  You can more easily make multipanel figures
\end{itemize}

We won't cover Divs in this workshop, but you can find out more
\href{https://quarto.org/docs/authoring/markdown-basics.html\#sec-divs-and-spans}{here}

\section{Tables}\label{tables}

Let's now add a table to our document!

There are several ways to add tables to a document. We will use the
\texttt{knitr::kable()} function to create a table from a data frame or
the summary information

🎬 Add a table of the summary information for the chaffinch data:

\begin{Shaded}
\begin{Highlighting}[]
\InformationTok{\textasciigrave{}\textasciigrave{}\textasciigrave{}\{r\}}
\InformationTok{\#| label: tbl{-}chaff}
\InformationTok{chaff\_summary |\textgreater{} }
\InformationTok{  knitr::kable(caption = "Summary of mass(g) for the two subspecies of chaffinch.", col.names = names(chaff\_summary) |\textgreater{} str\_to\_sentence())}
\InformationTok{\textasciigrave{}\textasciigrave{}\textasciigrave{}}
\end{Highlighting}
\end{Shaded}

\begin{verbatim}
Warning: 'xfun::attr()' is deprecated.
Use 'xfun::attr2()' instead.
See help("Deprecated")

Warning: 'xfun::attr()' is deprecated.
Use 'xfun::attr2()' instead.
See help("Deprecated")
\end{verbatim}

\begin{longtable}[]{@{}lrrrr@{}}

\caption{\label{tbl-chaff}Summary of mass(g) for the two subspecies of
chaffinch.}

\tabularnewline

\toprule\noalign{}
Subspecies & Mean & Sd & N & Se \\
\midrule\noalign{}
\endhead
\bottomrule\noalign{}
\endlastfoot
canariensis & 22.275 & 2.150857 & 20 & 0.4809462 \\
coelebs & 20.480 & 2.137780 & 20 & 0.4780222 \\

\end{longtable}

\subsection{Crossreferencing Tables}\label{crossreferencing-tables}

Just as the \texttt{fig-} prefix matters for corrected numbering of
figures, the \texttt{tbl-} prefix is important for the corrected
numbering of figures.

Just as the label of the code chunk is important for cross referencing
of figures so it is for tables and we use \texttt{@} followed by the
label of the code chunk to cross reference the table in the text.

🎬 If we add ``See \texttt{@tbl-chaff}'' to our text and hit Render
it'll display as:

See Table~\ref{tbl-chaff}.

\subsection{Tables - improve format}\label{tables---improve-format}

There's a lot of decimal places here that we can cut down. They can be
set in two ways:

\begin{enumerate}
\def\labelenumi{\arabic{enumi}.}
\item
  in the table by adding \texttt{digits\ =\ 2} to the \texttt{kable()}
  function
\item
  For the document overall by adding this to the chunk:
\end{enumerate}

\begin{Shaded}
\begin{Highlighting}[]
\InformationTok{\textasciigrave{}\textasciigrave{}\textasciigrave{}\{r\}}
\InformationTok{options(digits = 2)}
\InformationTok{\textasciigrave{}\textasciigrave{}\textasciigrave{}}
\end{Highlighting}
\end{Shaded}

🎬 Let's try the second option with this chunk:

\begin{Shaded}
\begin{Highlighting}[]
\InformationTok{\textasciigrave{}\textasciigrave{}\textasciigrave{}\{r\}}
\InformationTok{\#| label: tbl{-}chaff2}
\InformationTok{options(digits = 2)}
\InformationTok{chaff\_summary |\textgreater{} }
\InformationTok{  knitr::kable(caption = "Summary of mass(g) for the two subspecies of chaffinch.", col.names = names(chaff\_summary) |\textgreater{} str\_to\_sentence())}
\InformationTok{\textasciigrave{}\textasciigrave{}\textasciigrave{}}
\end{Highlighting}
\end{Shaded}

\begin{verbatim}
Warning: 'xfun::attr()' is deprecated.
Use 'xfun::attr2()' instead.
See help("Deprecated")

Warning: 'xfun::attr()' is deprecated.
Use 'xfun::attr2()' instead.
See help("Deprecated")
\end{verbatim}

\begin{longtable}[]{@{}lrrrr@{}}

\caption{\label{tbl-chaff2}Summary of mass(g) for the two subspecies of
chaffinch.}

\tabularnewline

\toprule\noalign{}
Subspecies & Mean & Sd & N & Se \\
\midrule\noalign{}
\endhead
\bottomrule\noalign{}
\endlastfoot
canariensis & 22 & 2.1 & 20 & 0.48 \\
coelebs & 20 & 2.1 & 20 & 0.48 \\

\end{longtable}

Note here we have two tables now, and the table numbering is automatic!

🎬 Cut the code chunk for the second table and Paste it above the first
table. Render, and observe how the Tables are automatically renumbered!

\section{The visual editor!}\label{the-visual-editor}

\begin{itemize}
\tightlist
\item
  The visual editor is a WYSIWYG editor for Quarto documents. It can be
  accessed by clicking the \texttt{Visual\ Editor} button in the top
  right or by Ctrl-Shift-F4
\item
  It makes it easier to add and edit content (although using source
  editor is often best for sanity checking!)
\item
  It is especially useful for adding citations, especially if you use
  zotero.
\end{itemize}

\section{Citations}\label{citations}

\subsection{Package citations}\label{package-citations}

You can add citations for R and packages, from doi, from a .bib file, or
from zotero.

🎬 Put your cursor where you want the citation to go and click the
\texttt{Insert} button in the top right and select \texttt{Citation}
(Ctrl-Shift-F8) and add a citation for the \texttt{tidverse} package.

🎬 Render

\begin{itemize}
\tightlist
\item
  the in-text citation has been added
\item
  the reference has been added to the references section
\item
  a \texttt{references.bib} file has been created in the project folder
\end{itemize}

\subsection{Citations from DOI}\label{citations-from-doi}

🎬 Put your cursor where you want the citation to go and click the
\texttt{Insert} button in the top right and select \texttt{Citation}
(Ctrl-Shift-F8) Select From DOI and paste in 10.1016/j.ympev.2009.07.018

🎬 Render

You can also add citations to the .bib file manually.

\section{Summary}\label{summary}

\begin{itemize}
\tightlist
\item
  Quarto is a multi-language scientific publishing system for producing
  dynamic, reproducible reports in many formats. It builds on the ideas
  of R Markdown.
\item
  The YAML header stores document metadata and controls default
  behaviour such as formatting and output options.
\item
  Code chunk options specify how chunks run and whether code and/or
  output appear in the rendered document.
\item
  Code can be executed interactively, allowing you to explore and refine
  analyses as you work.
\item
  Divs let you group sections of content and apply specific styles or
  layout structures to them.
\item
  Figures, tables, images, and equations can be automatically numbered
  and cross-referenced using labels (e.g., fig-, tbl-).
\item
  You can include citations and bibliographic references directly within
  your document.
\end{itemize}

\section{What Else Can I Do?}\label{what-else-can-i-do}

\begin{itemize}
\tightlist
\item
  Alter the formatting of your content with to create
  \href{https://quarto.org/docs/authoring/figures.html\#figure-panels}{panel
  figure} or
  \href{https://quarto.org/docs/authoring/figures.html\#caption-locations}{change
  figure legend locations}
\item
  Make a \href{https://quarto.org/docs/websites/}{website} with multiple
  pages\\
\item
  Make a \href{https://quarto.org/docs/presentations/}{presentation} in
  visualjs, pptx, or pdf format\\
\item
  Make a \href{https://quarto.org/docs/dashboards/}{dashboard} for
  interactive and informative data reporting\\
\item
  Create a
  \href{https://quarto.org/docs/manuscripts/index.html}{manuscript} for
  publication-ready papers
\item
  Integrate Google Sheets, create a book, embed interactive maps and
  lots, lots more! Check out the
  \href{https://quarto.org/docs/guide/}{Quarto guide} to find out more
  (and maybe get some inspiration!)
\end{itemize}

\textbf{Good luck on your Quarto adventures!}

Pages made with R (2024), Quarto (Allaire et al. 2022), knitr (Xie
2023), kableExtra (Zhu 2024)

\section{Want to gain more skills?}\label{want-to-gain-more-skills}

The UKRI Digital Research Skills Catalyst will be launching the Catalyst
Portal in early 2026!

Featuring:

\begin{itemize}
\tightlist
\item
  a huge repertoire of in-person and online training (both tutor-led and
  self-study)\\
\item
  a \textbf{free} 1-to-1 expert consultancy service\\
\item
  access to a calendar full of training and development events across
  the UK\\
\end{itemize}

Follow us on
\href{https://www.linkedin.com/company/ukri-digital-research-skills-catalyst/}{LinkedIn}
or
\href{https://bsky.app/profile/ukriskillscatalyst.bsky.social}{Bluesky}
to be informed as soon as the Portal launches!

\phantomsection\label{refs}
\begin{CSLReferences}{1}{0}
\bibitem[\citeproctext]{ref-allaire2022}
Allaire, J. J., Charles Teague, Yihui Xie, and Christophe Dervieux.
2022. {``Quarto,''} January.
\url{https://doi.org/10.5281/ZENODO.5960048}.

\bibitem[\citeproctext]{ref-base}
R Core Team. 2024. {``R: A Language and Environment for Statistical
Computing.''} \url{https://www.R-project.org/}.

\bibitem[\citeproctext]{ref-tidyverse}
Wickham, Hadley, Mara Averick, Jennifer Bryan, Winston Chang, Lucy
D'Agostino McGowan, Romain François, Garrett Grolemund, et al. 2019.
{``Welcome to the {\textbraceleft}Tidyverse{\textbraceright}''} 4: 1686.
\url{https://doi.org/10.21105/joss.01686}.

\bibitem[\citeproctext]{ref-knitr}
Xie, Yihui. 2023. {``Knitr: A General-Purpose Package for Dynamic Report
Generation in r.''} \url{https://yihui.org/knitr/}.

\bibitem[\citeproctext]{ref-kableExtra}
Zhu, Hao. 2024. {``kableExtra: Construct Complex Table with 'Kable' and
Pipe Syntax.''} \url{https://CRAN.R-project.org/package=kableExtra}.

\end{CSLReferences}




\end{document}
